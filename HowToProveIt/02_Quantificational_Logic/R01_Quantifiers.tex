We use \textit{quantifiers} to express how many values of a varialble make a statement true or if a statement is true for at least one value.

The symbol \(\forall\) is called the \textit{universal quantifier}, because the statement \(\forall x P(x)\) says that \(P(x)\) is \textit{universally} true.

The symbol \(\exists\) is called the \textit{existentail quantifer}, because the statement \(\exists x P(x)\) says that there is at least one value of \(x\) for which \(P(x)\) is true.

In general, if $x$ is a free variable in some statement $P(x)$, it is a bound variable in the satements $\forall x P(x)$ and $\exists x P(x)$. For this reason we say that the quantifiers \textit{bind} a variable. This means that a variable that is bound by a quantifier can always be replaced with a new variable without changing the meaning of the statement, and it is ofter possible to paraphrase the statemnt without mentioning the bound variable at all.
