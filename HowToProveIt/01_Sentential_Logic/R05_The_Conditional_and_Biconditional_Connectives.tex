The if-then logical connective, represented by the symbol \(\rightarrow\), is used to form conditional statements that expresses a relationship between the antecedent and its consequent.

The formulas \(P /rightarrow Q\), \(\lnot P \lor Q\) and \(\lnot (P \land \lnot Q)\)are equivalent.

The \textit{converse} of \(P \rightarrow Q\) is \(Q \rightarrow P\), its \textit{contrapositive} is \(\lnot Q \rightarrow \lnot P\) and its \textit{inverse} is \(\lnot P \rightarrow \lnot Q\)

The statement of the form \(P \iff Q\) is called a biconditional statement, and it is equivalent to \((P \rightarrow Q) \land (Q \rightarrow P)\)