When we evaluate the truth or falsity of a statement, we assign to it one of the labels \textit{true} or \textit{false}, and this label is called its \textit{truth value}.

A \textit{Truth table} is a table in which each of its rows shows one of the possible combinations of truth values for a statement or a compound statement.

To verify the validity of arguments, we can arrage the truth values of the premises and the conclusion in a truth table, in the rows where the premises are all true it must follow that the conclusion is also true, thus the argument is valid, otherwise it is invalid.

\textit{Equivalent} formulas always have the same truth value no matter what the truth value of those statements are.

Formulas that are always true, are called \textit{tautologies}, formulas that are always false are called \textit{contradictions}.