\begin{enumerate}
% 1
\item
\begin{enumerate}
    \item 
    Hypotheses: $n \in \mathbb{Z}$, $n > 1$, $n \text{ is not prime}$\\
    Conclusion: $2^n - 1 \text{ is not prime}$ \\
    - Yes the hypotheses are true when $n=6$
    - That $2^6 - 1$ is not prime
    - Yes $2^6 - 1 = 63$ is not prime.
    \item 
    $2^15 - 1 = 32767$ is not prime
    \item
    Nothing, $11$ is prime, thus one of the hypotheses is not true.
\end{enumerate}
% 2
\item
\begin{enumerate}
    \item 
    Hypotheses: $b^2 > 4ac$
    Conclusion: The quadratic equation $ax^2 + bx +c = 0$ has exactly two real solutions
    \item 
    Because x is not a free variable.
    \item 
    $a=2; b=-5; c=3; b^2 > 4ac \equiv (-5)^{2}>4(2)(3) \equiv 25 > 24$\\
    So $ax^2 + bx +c = 0$ has exactly two real solutions $x_1 = \frac{3}{2}; x_2 = 1$
    \item 
    $a=2; b=4; c=3; b^2 > 4ac \equiv (4)^{2}>4(2)(3) \equiv 16 > 24$\\
    Since the hypotheses is not true we cannot conclude anything from the theorem.
\end{enumerate}
% 3
\item
Hypotheses: $n$ is a natural number larger than $2$, $n$ is not a prime number
Conclusion: $2n + 13$ is not a prime number
Counterexample: $n=8; 2(8)+13 = 29$ which is not a prime number
% 4
\item
\textit{Proof}. Suppose $0 < a < b$. Then $b - a > 0$.
Multiplying $b - a > 0$ by $b + a$, we obtain $(b+a) \cdot (b-a) > (b+a) \cdot 0$ which is the same as $b^2 - a^2 > 0$.
Since $b^2 - a^2 > 0$, it follows that $a^2 < b^2$. Therefore if $0 < a < b$ then $a^2 < b^2$
% 5
\item
\textit{Proof}. Supposing that $a<b<0$.
Then multiplying the negative number $a$ into the inequality $a<b$, we obtain $a^2 > ab$, and now multiplying the negative number $b$ into the inequality $a<b$, we obtain $ab > b^2$.
Since $a^2 > ab > b^2$, it follows that $a^2 > b^2$. Therefore if $a<b<0$ then $a^2 > b^2$
% 6
\item
\textit{Proof}. Suppose that $0 < a < b$. Then multiplying the inequality $a<b$ by $\frac{1}{ab}$, we obtain $\frac{1}{b} < \frac{1}{a}$. Thus if $0 < a < b$ then $\frac{1}{b} < \frac{1}{a}$
% 7
\item
\textit{Proof}. Suppose that $a^3 > a$.
Multiplying $a^2$ to $a^3 > a$ we get $a^5 > a^3$. Since $a^5 > a^3 > a$, it follows that $a^5 > a$. Therefore if $a^3 > a$ then $a^5> a$.
% 8
\item
\textit{Proof}. Suppose that $A \setminus B \subseteq C \cap D$ and $x \in A$ and $x \notin D$. Then if $x \notin B$ and $x \in A$ then $x \in A \setminus B$. Since $A \setminus B \subseteq C \cap D$ then $x \in C \cap D$ thus $x \in C \land x \in D$, Therefore $x \in D$.
Thus if $x \in D$ then $x \notin B$
% 9
\item
\textit{Proof}. Suppose that $A \cap B \subseteq C \setminus D$ and $x \in A$. If $x \in B$, then since $x \in A$ means that $x \in A \cap B$, since $A \cap B \subseteq C \setminus D$ then $x \in C \setminus D$ thus $x \in C \land x \notin D$. 
Therefore if $x \in B$ then $x \notin D$
% 10
\item
\textit{Proof}. Suppose that $a$ and $b$ are real numbers. If $\frac{a+b}{2} \ge b$. Then, $a+b \ge 2b$ and $a \ge b$. Therefore if $\frac{a+b}{2} \ge b$ then $a \ge b$
% 11
\item
\textit{Proof}. Suppose that $x = 8$. Then $\frac{\sqrt[3]{x} + 5}{x^2 + 6} = \frac{\sqrt[3]{8} + 5}{8^2 + 6} = \frac{2 + 5}{64 + 6} = \frac{7}{70} = \frac{1}{7} \neq \frac{1}{x} = \frac{1}{x}$. Thus if $x = 8$ then $\frac{\sqrt[3]{x} + 5}{x^2 + 6} \neq \frac{1}{x}$
% 12
\item
\textit{Proof}. Suppose that $ac \ge bd$. Then dividing $ac \ge bd$ by $d$, $\frac{ac}{d} \ge b$. Combining $\frac{ac}{d} \ge b$ with $a < b$, we obtain $a < b \le \frac{ac}{d}$, thus $a < \frac{ac}{d}$ and $1 < \frac{c}{d}$ so $c > d$
% 13
\item
\textit{Proof}. Suppose that $x > 1$, then simplifying $3x + 2y \leq 5$ for $x$, we obtain that $x \leq \frac{5 - 2y}{3}$, so $1 < x < \frac{5 - 2y}{3}$, then $1 < \frac{5 - 2y}{3}$, simplifying for $y$ we obtain that $y < 1$
% 14
\item
\textit{Proof}. Suppose that $x^2 + y = -3$ and that $2x - y = 2$. Adding the equiations we obtain $x^2 + 2x = -1$, rewriting as $(x+1)^2 = 0$, we see that $x = -1$
% 15
\item
\textit{Proof}. Suppose that $x > 3$ and $y < 2$.
Thus $x^2 > 9$ and $-2y > -4$, adding the inequalities we obtain $x^2 - 2y < 5$
% 16
\item
\begin{enumerate}
    \item We are supposing that the conclusion it's true, we cannot infer anything from the hypotheses.
    \item 
\textit{Proof}. Supposing that $\frac{2x-5}{x-4}=3$, then $(2x-5)=3(x-4)$, simplifying $-x+7=0$, which means that $x = 7$
\end{enumerate}
% 17
\item
\begin{enumerate}
    \item The other possible value of $x$ that can make the equation true is being ignored.
    \item $x = -3; y = 1$
\end{enumerate}
\end{enumerate}

