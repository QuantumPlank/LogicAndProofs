Mathematicians usually state the answer to a mathematical question in the form of a \textit{theorem} that says that if certain assumptions called \textit{hypotheses} of the theorem are true, then some conclusion must also be true.

Often the hypotheses and conclusion contain free variables, and in this case it isunderstood that these variables can stand for any elements of the universe of discourse. An assignment of particular values to these variables is called an \textit{instance} of the theorem, and in order for the theorem to be correct it must be the case that for every instance of the theorem that makes the hypotheses come out true, the conclusion is also true. If there is even one instance in which the hypotheses are true but the conclusion is false, then the theorem is incorrect. Such an instance is called a \textit{counterexample} to the theorem.

If a counterexample for a theorem is found then we can be sure that the theorem is incorrect, but the only way to know that a theorem is correct is to prove it. A proof of a theorem is simply a deductive argument whose premises are the hypotheses of the theorem and whose conclusion is the conclusion of the theorem. Throught the proof, we think of any free variables in the hypotheses and conclusion of the theorem as standing for some particular but unspecified elements of the universe of discourse.

\textit{Never assert anything until you can justify it completely}. This is the primary purpose of any proof: to provide a guarantee that the conclusion is true if the hypotheses are.

We will refer to the statements that are known or assumed to be true at some point in the course of figuring out a proof as \textit{givens}, and the statemtn that remains to be provenn at a point as the \textit{goal}. 

When we are starting to figure out a proof, the givens will be just the hypotheses of the theorem we are proving, but they may later include other statements that have been inferred from the hypotheses or added as new assumptions as the result of some transformation of the problem. The goal will initially be the conclusion of the theorem, but it may be changed several times in the course of figuring out a proof.

\textbf{To prove a goal of the form} $P \rightarrow Q$:\\
Assume $P$ is true and then prove $Q$
